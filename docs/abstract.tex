\documentclass[10pt]{article}
\usepackage[margin=0.5in]{geometry}
\usepackage{hyperref}

\begin{document}
\title{Performance analysis and optimization of C$++$ standard libraries}
\date{}

\maketitle

\section*{Summary}
C++ programs are widely used in performance critical applications. The standard
libraries of C++, hence, are expected to be very efficient. However,
experimental results show opportunities for improvements in some of the most
commonly used data structures and algorithms.

We will present the performance analysis work on libc$++$ and libstdc$++$ and
the changes we did to these libraries and to the LLVM compiler to optimize the
code using them. This includes our contributions to standard library algorithms
like string::find, libc$++$::basic\_streambuf::xsgetn, and libc$++$::locale. We
improved these suboptimal algorithms, particularly string::find which improved
by more than 10x. Similarly, we enabled the inlining of constructor and
destructor of libc$++$::string. We will present a systematic analysis of
function attributes and the places where we added missing attributes. We will
present a comparative analysis of libc$++$ vs. libstdc$++$ vs. Microsoft's C++
standard library on commonly used data structures and algorithms based on our
std-benchmark (https://github.com/hiraditya/std-benchmark), that we started
developing to help analyze standard C$++$ libraries. We will discuss the
performance issues with libc$++$::stringstream and libc$++$::sort that we are
currently working on. We will also present the lessons learned as a result of
analyzing C$++$ standard libraries, for example:
\begin{enumerate}
\item Iterator based algorithms can lose information and hence, can result in
  suboptimal performance.  This is exemplified in the implementation of
  std::rotate where we can just exchange few pointers should the underlying
  container is a doubly linked list e.g., std::list.
\item The C++ programming language has a limitation that the constructor and
  destructor cannot be const qualified which could have facilitated useful
  compiler optimizations like removing the destructor of a const std::string
  when the string is small enough to be kept on the stack.
\end{enumerate}

Keywords: C++, performance analysis, benchmarking libraries, compiler
optimization, LLVM, libstc$++$, libc$++$
\\
\\
Reference to previous talks: \url{http://sched.co/A8J7}, \url{http://sched.co/8Yzk}

\end{document}
