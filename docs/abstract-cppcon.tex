\documentclass[10pt]{article}
\usepackage[margin=0.5in]{geometry}
\usepackage{hyperref}
\usepackage{comment}
\usepackage{amsmath}
\usepackage{graphicx}
\usepackage{amssymb}
\usepackage{graphviz}
\usepackage{auto-pst-pdf}
\usepackage{etoolbox}
\usepackage{flushend}
\usepackage{needspace}
%\usepackage{authblk}


\begin{document}
\title{Performance analysis and optimization of system libraries}

%\author[1]{Aditya Kumar}
%\author[2]{Sebastian Pop}
%\affil[1]{Samsung Austin R\&D Center}
%\affil[2]{Samsung Austin R\&D Center}
%\affil[ ]{\textit {\{aditya.k7, s.pop}\}@samsung.com}

\date{}
\maketitle

\section*{Summary}
C/C$++$ programs are widely used in performance critical applications, as such,
they are expected to be very efficient. However, experimental results show
opportunities for improvements in some of the most commonly used data structures
and algorithms.

We will present the performance analysis work on widely used system libraries
like libc$++$, libstdc$++$, zlib, libziparchive (Android core), and the changes
we did to these libraries and to the GCC/LLVM compiler to optimize them. This
includes our contributions to standard library algorithms like string::find,
libc$++$::basic\_streambuf::xsgetn, and libc$++$::locale. We improved these
suboptimal algorithms, particularly string::find which improved by more than
10x. Similarly, we enabled the inlining of constructor and destructor of
libc$++$::string. We also improved the hottest algorithms in zlib and
libziparchive by several factors. We will highlight useful optimization tricks
that we used as part of optimizing these libraries.

We will present a systematic analysis of C++ standard libraries which enabled us
to expose differences in their design as well as their dynamic behavior. We will
present a comparative analysis of libc$++$ vs. libstdc$++$ vs. Microsoft's C++
standard library on commonly used data structures and algorithms based on our
std-benchmark (https://github.com/hiraditya/std-benchmark), that we started
developing to help analyze standard C$++$ libraries. We will discuss the
performance issues with libc$++$::stringstream and libc$++$::sort that we are
currently working on. We will also present the lessons learned as a result of
analyzing C$++$ standard libraries, for example:
\begin{enumerate}
\item Iterator based algorithms can lose information and hence, can result in
  suboptimal performance.  This is exemplified in the implementation of
  std::rotate where we can just exchange few pointers and avoid several useless
  copies should the underlying container be a doubly linked list e.g.,
  std::list.
\item The C++ programming language has a limitation that the constructor and
  destructor cannot be const qualified which could have facilitated useful
  compiler optimizations like removing the destructor of a const std::string
  when the string is small enough to be kept on the stack. This problem was
  pointed out a long time back but there seems to be no follow up
  (wg21/docs/papers/1995/N0798.htm).
\end{enumerate}

Keywords: C++, performance analysis, benchmarking libraries, compiler
optimization, GCC, LLVM, libstc$++$, libc$++$
\\
\\
Reference to previous talks: \url{http://sched.co/A8J7}, \url{http://sched.co/8Yzk}

\end{document}
