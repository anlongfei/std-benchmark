\documentclass{sig-alternate}
\usepackage{comment}
\usepackage{amsmath}
\usepackage{hyperref}
\usepackage{graphicx}
\usepackage{amssymb}
\usepackage{graphviz}
\usepackage{auto-pst-pdf}
\usepackage{etoolbox}
\usepackage{flushend}
\usepackage{needspace}

\makeatletter
\preto{\@verbatim}{\topsep=1pt \partopsep=0pt}
\makeatother

\pagenumbering{arabic}

\begin{document}
\def \GCC {GCC}
\def \LLVM {LLVM}

\special{papersize=8.5in,11in}
\setlength{\pdfpageheight}{\paperheight}
\setlength{\pdfpagewidth}{\paperwidth}

\title{Std-benchmarking}

\toappear{
   \hrule \vspace{5pt}
   Some conf
}
\numberofauthors{3}

\author{
\alignauthor
Aditya Kumar\\
       \affaddr{Samsung Austin R\&D Center}\\
       \email{aditya.k7@samsung.com}
\and
\alignauthor
Sebastian\\
       \affaddr{Samsung Austin R\&D Center}\\
       \email{ancd}
}

\maketitle
\begin{abstract}
We present a systematic analysis of C and C++ standard libraries. The goal
here is to enable each programmer make informed decision about the functionality
one is using and not just rely on common wisdom. We make is very easy to know
the internals of the standard libraries. There are benchmark analyses available
online but those are in bits and pieces. We present a comprehensize infrastructure
where a large set of standard library is covered. This also enables someone
to run their own configuration of test by adding just few lines of code. This also
allows the programmer to compare which compiler toolchain generates better code
in terms of performance, code-size etc. such that they can choose the right
toolchain for their application. The comparative analysis of multiple toolchains
also enabled us to improve the slow library functions.
\end{abstract}

\section{Introduction}
Why a systematic analysis is important.

The main contributions of this paper are:
\begin{itemize}
\item a benchmark suite for C/C++ standard library
\item ability to compare compiler performance for standard libraries
\item identifying slower implementation in standard library
\item investigating whether C++11/14 really makes your code faster (for
standard libraries) at -O0, -O3
\end{itemize}

\section{Related Work}
Several bits and pieces of benchmarking available online.
Bjarne's channel9 talk \cite{stroustrup2012}.
clrs \cite{clrs}

\newpage

\subsection{Layout of the project}
Structure

\subsection{Illustrative Example} \label{subsec:example}
How to add a single benchmark \cite{googlebench}
\newpage

\section{Experimental Results and discussion}
\subsection{Benchmark results comparison across toolchains}

\subsection{Time complexity results}

\subsection{C vs C++ algorithms}
string::find vs. strstr.

We present the results we got on x86-64 as well as aarch64 machines.

\section{Timing and Limitations}
Timing and Limitations
\subsection{Limitations of the time-complexity measurement}

\section{Conclusion and Future Work}

\bibliographystyle{abbrv}
{\small
\bibliography{Bibliography}
}
\end{document}
